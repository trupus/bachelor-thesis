\chapter{Introduction}
\label{ch:Introduction}

One of the most occurring temporomandibular joint (TMJ) dysfunction is bruxism. Altough debatable some studies suggest prelevance in up to 15\% of children and in as many as 96\% of adults \cite{thompson1994treatment}.

It can manifest itself trhoughout the day or night or both. 

We diferentiate between ``Awake Bruxism`` (AB) or ``Sleep Bruxism`` (SB) based on the time-period when it occurs. The masticatory muscle contraction and contraction types are also different from person to person and between AB and SB

In esence bruxism is a set of patterns of movements of the jaw with different periodicity, force and duration. Some of such patterns include clenching, grinding and sliding of the jaw.

Bruxism can be hard to diagnose

80\% of bruxism episodes are not accompanied by noise \cite{shetty2010bruxism}

80\% of bruxers are not aware of their condition \cite{thompson1994treatment}

If not treated bruxism can cause to teeth fracture, headache or masticatory muscle problems.

The ultimative diagnose of the bruxism is usually done in a controlled lab environment during a sleep study. This tends to be very expensive and impractical in assesting of the AB.

A more convinient and universal way would be the use of one or a combination of affordable wearable devices. The primary purpose of such wearables can be completly different, but the internal sensors can be repurposed and trained to classify jaw movements.

Based on previous research in the field, it was interesting to compare different wearables in the same experiment environment.