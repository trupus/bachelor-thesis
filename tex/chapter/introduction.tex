\chapter{Introduction}
\label{ch:Introduction}

Cheap and user-friendly wearable devices are needed to diagnose and prevent temporomandibular joint dysfunctions (TMD). Bruxism is one of the most occurring of such dysfunctions. In essence, it is a set of movement patterns of the jaw with different periodicity, force, and duration that occur involuntarily. These patterns include clenching, grinding, and sliding of the jaw. We differentiate between ``Awake Bruxism`` (AB) and ``Sleep Bruxism`` (SB) based on the time-period when it occurs. The masticatory muscle contraction and contraction types are also different from person to person and between AB and SB.

This presents a challenge in identifying patients that are experiencing one of the forms of bruxism, but are unaware of it because no damage has been done yet. 80\% of bruxism episodes are not accompanied by noise \cite{shetty2010bruxism}. 80\% of bruxers are not aware of their condition \cite{thompson1994treatment}. If left untreated it may cause teeth fracture, headache, or masticatory muscle problems.

There isn't a clear consensus about the prevalence of bruxism in the human population, as the numbers are fluctuating around 14 to 20\% in children and 16 to 96\% in the adult population \cite{manfredini2013epidemiology}, \cite{thompson1994treatment}. The methods for the diagnosis present today are either unreliable or invasive and expensive.

A more convenient and universal way would be the use of one or of a combination of affordable wearable devices. The primary purpose of such wearables can be completely different (e.g. earbuds), but the sensors present on them can be repurposed and trained to classify jaw movements.

Based on previous research in the field, it was interesting to see how different wearables would compare in the same experiment environment. A custom-built prototype was evaluated in a user study. The user study was designed with different motor activities (relevant for AB and SB) in mind.