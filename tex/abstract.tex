\chapter*{Abstract}
\label{ch:Abstract}

Bruxism is one of the most occurring temporomandibular joint dysfunctions (TMD) characterized by contractions or contraction bursts of masticatory muscles. 80\% of bruxers don't even know they have it. If left untreated it can lead to headache, masticatory muscle problems, or teeth fractures. We need cheap, user-friendly, and reliable tools to diagnose and prevent it, as the currently available diagnosis methods tend to be expensive and invasive. In the scope of this study 5 devices were selected and combined together in a custom-built prototype. A user study was designed to simulate bruxism with clenching, grinding, and sliding, as well as comparing it to contrast jaw movements like reading, drinking, chewing, smiling, etc. Traditional machine learning approaches were used in a brute-force approach to classifying bruxism. Four classifiers were mapped with a total of 255 groups of time-series and trained, and an accuracy score was computed for every one of them. This yielded a big sortable table that can be used as a reference for future research. The selection criteria discussed here was to have the highest accuracy while keeping the lowest amount of used time series. We observe that the time series of quaternions from two IMUs placed on the back side of the jaw paired with the random forest classifier, fit these selection criteria best, achieving 88\% accuracy in identifying bruxism.