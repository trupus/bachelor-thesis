\pgfplotsset{compat=newest}

% Settings for Titlepage
\hypersetup{
	colorlinks,
  citecolor=black,
  filecolor=black,
  linkcolor=black,
  urlcolor=black,
 pdfauthor={FirstName LastName},
 pdftitle={Title of Thesis}
 pdfsubject={Not set},
 pdfkeywords={Not set}
}
% \DeclareFloatingEnvironment[fileext=frm,placement={!ht},name=Listing,within=section]{listing}

% Print URLs not in Typewriter Font
\def\UrlFont{\rm}

\newcommand{\specialcell}[2][c]{%
  \begin{tabular}[#1]{@{}c@{}}#2\end{tabular}}

\newcommand\todo[1]{\textcolor{red}{TODO: #1}}

\newcommand\hlcode[1]{\textcolor{red}{#1}}

\newcommand\citeable[1]{\textcolor{green}{\hl{citeable: #1}}}

\newcolumntype{\$}{>{\global\let\currentrowstyle\relax}}
\newcolumntype{^}{>{\currentrowstyle}}
\newcommand{\rowstyle}[1]{\gdef\currentrowstyle{#1}%
  #1\ignorespaces
}

% Comments
\newif\ifcomment
%\commenttrue

% Leerseite ohne Seitennummer, nächste Seite rechts
\newcommand{\blankpage}{
 \clearpage{\pagestyle{empty}\cleardoublepage}
}

% \lstset{%
%   language = Octave,
%   backgroundcolor=\color{white},   
%   basicstyle=\footnotesize\ttfamily,       
%   breakatwhitespace=false,         
%   breaklines=true,                 
%   captionpos=b,                   
%   commentstyle=\color{gray},    
%   deletekeywords={...},           
%   escapeinside={\%*}{*)},          
%   extendedchars=true,              
%   frame=single,                    
%   keepspaces=true,                 
%   keywordstyle=\color{orange},       
%   morekeywords={*,...},            
%   numbers=left,                    
%   numbersep=5pt,                   
%   numberstyle=\footnotesize\color{gray}, 
%   rulecolor=\color{black},         
%   rulesepcolor=\color{blue},
%   showspaces=false,                
%   showstringspaces=false,          
%   showtabs=false,                  
%   stepnumber=2,                    
%   stringstyle=\color{orange},    
%   tabsize=2,                       
%   title=\lstname,
%   emphstyle=\bfseries\color{blue}%  style for emph={} 
% } 

% %% language specific settings:
% \lstdefinestyle{Arduino}{%
%     language = Octave,
%     keywords={void, int boolean},%                 define keywords
%     morecomment=[l]{//},%             treat // as comments
%     morecomment=[s]{/*}{*/},%         define /* ... */ comments
%     emph={HIGH, OUTPUT, LOW}%        keywords to emphasize
% }